%%%%%%%%%%%%%%%%%%%%%%%%%%%%%%%%%%%%%%%%%%
%  file: lectureNotesExample.sty
%  author: Kyle G. Burke
%  description: An example using the commands from the lectureNotes.sty package.
%%%%%%%%%%%%%%%%%%%%%%%%%%%%%%%%%%%%%%%%%%
\documentclass[letter,12pt]{article}
\usepackage[utf8x]{inputenc}
\usepackage{lectureNotes}

%sets whether this is For students or not.  one of these needs to be set!
\togglefalse{forStudents} %use this option to leave the answers in for your own purposes
%\toggletrue{forStudents} %use this to remove the answers to questions

%opening
\lectureNotesTitle{CS 200: Faking It}
\author{Kyle Burke}

\begin{document}

\maketitle

Welcome to the class!

\lectureAction{Go over Syllabus.}

\lecturePoint{Squares are really awesome objects.  I love them!}

\question{Does anyone else love squares?}

%notice that the answer comes first, since it's an optional argument.
\question[Four.]{How many sides are in a square?}

\giveawayq[All okay: Rectangle, Rhombus, Parallelogram, Quadrillateral, Kite.]{What's another kind of shape with four sides?}

\lectureDefinition{Square}{A \emph{square} is a polygon with four right angles and four sides of equal length.}

\lectureExample{Square}{The following points are corners of a square: $(0, 0)$, $(2, 0)$, $(0, 2)$, $(2, 2)$.}

\lecturePoint{The following code could be used to create a Square object in Java:}

\codeExample{int sideLength = 4;\\
Square square = new Square(sideLength);}

\hwAssign{Chapter 0, Section 0, Problems: 0, 3, and 5.}

\end{document}
