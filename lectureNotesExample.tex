%%%%%%%%%%%%%%%%%%%%%%%%%%%%%%%%%%%%%%%%%%
%  file: lectureNotesExample.sty
%  author: Kyle Burke <paithanq@gmail.com>
%  description: An example using the commands from the lectureNotes.sty package.
%%%%%%%%%%%%%%%%%%%%%%%%%%%%%%%%%%%%%%%%%%
\documentclass{article}
\usepackage[utf8x]{inputenc}
\usepackage{amsmath}
\usepackage{url}
\usepackage{hyperref}
\usepackage{lectureNotes}

%sets whether this is For students or not.  one of these needs to be set!
\togglefalse{forStudents} %use this option to leave the answers in for your own purposes
%\toggletrue{forStudents} %use this to remove the answers to questions

\textwidth 6.5in
\oddsidemargin 0.0in
\evensidemargin 0.0in

%opening
\lectureNotesTitle{CS 200: Faking It}
\lectureNotesAuthor{Kyle Burke}

\begin{document}

\maketitle

Welcome to the class!

\lectureAction{Go over Syllabus.}

\lecturePoint{Squares are really awesome objects.  I love them!}

\question{Does anyone else love squares?}

%notice that the answer comes first, since it's an optional argument.
\question[Four.]{How many sides are in a square?}

\giveawayq[All okay: Rectangle, Rhombus, Parallelogram, Quadrillateral, Kite.]{What's another kind of shape with four sides?}

\lectureDefinition{Square}{A \emph{square} is a polygon with four right angles and four sides of equal length.}

\lectureExample{Square}{The following points are corners of a square: $(0, 0)$, $(2, 0)$, $(0, 2)$, $(2, 2)$.}

\lecturePoint{The following code could be used to create a Square object in Java:}

\codeExample{int sideLength = 4;\\
Square square = new Square(sideLength);}

%Exercises!
\beginExercisesPart

\exerciseVisibleAnswer{squaresQ}{How many sides does a square have?}{A square has four sides.}

\exerciseInstructorAnswer{harderSquaresQ}{If one side of a square has length 1, what is the total perimeter of that square?}{
    \begin{align*}
        \text{Perimeter} &= 1 + 1 + 1 + 1\\
        &= 2 + 1 + 1\\
        &= 3 + 1\\
        &= 4
    \end{align*}
    The total perimeter of the square is 4.
}

%I don't like this command and should deprecate it...
\hwAssign{Chapter 0, Section 0, Problems: 0, 3, and 5.}

\subsection{Acknowledgements}

Thanks to all the students I've had who have put up with annoying versions of these notes.  Sorry!  Thanks for persisting through it!

Thanks also to the authors of the \LaTeX\ exercise package\footnote{\url{https://www.ctan.org/tex-archive/macros/latex/contrib/exercise}}.  This made it possible for me to create the commands that add exercises.  Extremely helpful!

I got lots of help from the \TeX\ stack exchange.  Thanks to all the people who responded to my questions about using environments inside  a phantom command\footnote{\url{http://tex.stackexchange.com/questions/229411/can-i-use-phantom-to-hide-a-latex-environment-itemize}} and about how to use optional parameters in commands\footnote{\url{http://tex.stackexchange.com/questions/84595/latex-optional-arguments-with-square-brackets}}.


%%%%%%%%%%%%%%%%%%%%% appendix! %%%%%%%%%%%%%%%%%%%%%%%%%
\appendix

%%%%%%%%%%%%%%%%%%%%% Answers to Exercises! %%%%%%%%%%%%%
\section{Answers to Exercises}

\shipoutAnswer

\end{document}
